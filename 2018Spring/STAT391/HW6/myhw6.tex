\documentclass[preprint,12pt]{elsarticle}

    \usepackage[sc]{mathpazo} % Use the Palatino font
    \usepackage[T1]{fontenc} % Use 8-bit encoding that has 256 glyphs
    \usepackage{microtype} % Slightly tweak font spacing for aesthetics
    \usepackage[english]{babel} % Language hyphenation and typographical rules
    \usepackage{booktabs} % Horizontal rules in tables
    \usepackage{enumitem} % Customized lists
    \usepackage[table,xcdraw]{xcolor}
    \usepackage[utf8]{inputenc} % Required for inputting international characters
    \usepackage{parskip}
    \usepackage{graphicx}
    \usepackage{hyperref}
    \usepackage{pdfpages}
    \usepackage{amsmath}
    \usepackage{esvect}
    \usepackage{listings}
    \usepackage{color}
    \usepackage{spverbatim}
    \usepackage{subcaption}
    \usepackage{multirow}
    \usepackage[title]{appendix}
    \hypersetup{
        colorlinks=true,
        linkcolor=blue,
        filecolor=magenta,      
        urlcolor=cyan,
    }
    \definecolor{dkgreen}{rgb}{0,0.6,0}
    \definecolor{gray}{rgb}{0.5,0.5,0.5}
    \definecolor{mauve}{rgb}{0.58,0,0.82}
    \definecolor{lightgray}{rgb}{0.83, 0.83, 0.83}
    \definecolor{timberwolf}{rgb}{0.86, 0.84, 0.82}
    \definecolor{whitesmoke}{rgb}{0.96, 0.96, 0.96}
    
    \lstset{frame=tb,
    language=python,
    aboveskip=3mm,
    belowskip=3mm,
    showstringspaces=false,
    columns=flexible,
    basicstyle={\small\ttfamily},
    numbers=none,
    numberstyle=\tiny\color{gray},
    keywordstyle=\color{blue},
    commentstyle=\color{dkgreen},
    stringstyle=\color{mauve},
    breaklines=true,
    breakatwhitespace=true,
    tabsize=3,
    backgroundcolor = \color{whitesmoke}
    }

    \begin{document}
    \title{\LARGE \bf
        STAT 391 Homework 6
        }
        
        \author{ \parbox{3 in}{\centering Chongyi Xu \\
                 University of Washington\\
                 STAT 391 Spring 2018\\
                 {\tt\small chongyix@uw.edu}}
        }
    \maketitle

    \section{Problem 1 - Statistical decision making}
    \begin{enumerate}[label=\alph*]
        \item Make a neatly labeled table of the outcome space for
        this problem
        \begin{table}[]
			\centering
			\begin{tabular}{|l|l|}
			\hline
			1st Day     & 2nd Day             \\ \hline
			F,G|A     & --                  \\ \hline
			F,G|B     & --                    \\ \hline
			$\bar{F}$,G & F,G                 \\ \hline
						& F,$\bar{G}$         \\ \hline
						& $\bar{F}$,G         \\ \hline
						& $\bar{F}$,$\bar{G}$ \\ \hline
			\end{tabular}
			\caption{Outcome Space}
			\label{table1}
		\end{table}
		Table\ref{table1} is the outcome space.

        \item What is the probability that Rob finds the graphics card
        on the second day?
        \begin{align*}
			P(F\text{ on the second day}) &= P(\bar{F}|B) * P(F|A) * P(A) + P(\bar{F}|A) * P(F|B) * P(B)\\
			&= (1-\frac{2}{5}) \cdot \frac{1}{2} \cdot \frac{1}{3} + \frac{1}{2}\cdot \frac{2}{5}\cdot \frac{2}{3}\\
			&= \frac{1}{10} + \frac{2}{15}\\
			&= \frac{7}{30}
        \end{align*}

		\item What is the probability that he finds the graphics card?
		\begin{align*}
			P(F) &= P(F\text{ on the first day}) + P(F\text{ on the second day})\\
			&= P(F|B) * P(B) + P(F|A) * P(A) + \frac{7}{30}\\
			&= \frac{4}{15} + \frac{1}{6} + \frac{7}{30}\\
			&= \frac{2}{3}
		\end{align*}
		
		\item What is the probability that he finds the card and the card
		is still good?
		\begin{align*}
			P(F,G) &= P(F|B) * P(B) + P(F|A) * P(A) + P(\bar{F}|B) * P(F|A) * P(A) * P(G|A) + P(\bar{F}|A) * P(F|B) * P(B) * P(G|B)\\
			&= \frac{4}{15} + \frac{1}{6} + \frac{1}{10} * \frac{4}{5} + \frac{2}{15} * \frac{3}{5}\\
			&= \frac{89}{150} 
		\end{align*}

		\item What is the expected value of Rob's search policy?
		Consider the outcome space

		
	\end{enumerate}
	
	\section{Problem 3- Bayesian Inference}
	\begin{enumerate}[label=\alph*]
		\item Compute the probability that a customer buys all three
		books under $\bar{P}_{ABC}$.
		\begin{equation*}
			\bar{P}_{ABC}(1,1,1) = 0.6\cdot 0.3\cdot 0.4 = 0.072
 		\end{equation*}
		 
		\item Compute the probability that a customer buys all three
		books under $\tilde{P}$
		\begin{equation*}
			\tilde{P}_{ABC}(1,1,1) = 0.1\cdot 0.4 = 0.04
		\end{equation*}
		 
		\item Using $\bar{P}$ and $\tilde{P}$ from above, determine if 
		the likelihood that Robin Hood is a man is higher than the 
		likelihood that the (s)he's a woman.
		\begin{align*}
			\bar{P}_{ABC}(1,1,0) &= 0.6\cdot 0.3\cdot (1-0.4) = 0.108
			\tilde{P}_{ABC}(1,1,0) &= 0.1\cdot (1-0.4) = 0.06
		\end{align*}
		Therefore, the likelihood tells Robin Hood is more likely to be 
		a woman.

		\item Determine the posterior probability that Robin Hood is a man.
		\begin{align*}
			P(man|A=1,B=1,C=0) &= \frac{P((1,1,0)|man)P(man)}{P(1,1,0)}\\
			&= \frac{0.06\cdot \frac{2}{3}}{0.06\cdot \frac{2}{3}+0.108\cdot \frac{1}{3}}\\
			&= \frac{0.04}{0.04+0.036}\\
			&= \frac{10}{19}
		\end{align*}

		\item Determine the posterior probability that RObin Hood is a man if 
		Al doesn't recall whether Robin ordered Book C or not.
		\begin{align*}
			P((1,1)|woman) &= \bar{P}_{AB}(1,1) = 0.6\cdot 0.3 = 0.18\\
			P((1,1)|man) &= \tilde{P}_{AB}(1,1) = 0.1\\
			P(man|A=1,B=1) &= \frac{P((1,1)|man)P(man)}{P(1,1)}\\
			&= \frac{0.1\cdot \frac{2}{3}}{0.18\cdot \frac{1}{3} + 0.1\cdot \frac{2}{3}}\\
			&= \frac{10}{19}
		\end{align*}

		\item Compute the value of Likelihood Ratio(LR) for the data $A=1,B=1,C=0$
		\begin{equation*}
			LR(A=1,B=1,C=0) = \frac{\bar{P}_{ABC}(1,1,0)}{\tilde{P}_{ABC}(1,1,0)} = \frac{0.108}{0.06} = 1.8
		\end{equation*}

		Compute the value of the LR if the data consists of 3 customers 
		$A_1=1,B_1=0,C_1=0,A_2=0,B_2=1,C_2=0,A_3=1,B_3=0,C_3=1$.
		\begin{align*}
			LR(D) &= \frac{\bar{P}_{ABC}(1,0,0)\cdot \bar{P}_{ABC}(0,1,0)\cdot \bar{P}_{ABC}(1,0,1)}{\tilde{P}_{ABC}(1,0,0)\cdot \tilde{P}_{ABC}(0,1,0)\cdot \tilde{P}_{ABC}(1,0,1)}\\
			&= \frac{0.6*(1-0.3)*(1-0.4)*(1-0.6)*0.3*(1-0.4)*0.6*(1-0.3)*0.4}{0.5*(1-0.4)*0.2*0.4*0.5*0.4}\\
			&= 0.63504
		\end{align*}

		\item Give an example of a data set where $LR>1$.
		\begin{equation*}
			D:A_1=1,B_1=1,C_1=0,A_2=1,B_2=1,C_2=1
		\end{equation*}
		
	\end{enumerate}

	\section{Problem 5 - Dirichlet/Beta Distribution}
	\begin{enumerate}[label=\alph*]
		\item Change the variables $\theta_j$ to $\xi_j=ln \theta_j$ and express
		$L$ as a function of $\xi$.
		\begin{equation*}
			L = P(\theta_1,\theta_2) = \theta_1^{n_1}\theta_2^{n_2} = n_1\xi_1 n_2\xi_2
		\end{equation*}

		
	\end{enumerate}

\end{document}