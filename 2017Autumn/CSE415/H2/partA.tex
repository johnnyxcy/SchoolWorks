\documentclass[]{exam}
    \usepackage[utf8]{inputenc}
    \usepackage{enumitem} % allows us to use the enumerate command
    \usepackage{amsmath} % allows us to type in math
    \usepackage{amsfonts}
    \usepackage{setspace}
    \usepackage{verbatim} % allows us to type code like text
    \usepackage{graphicx} % allows us to include figures
    \usepackage{gensymb}
    \usepackage{color}
    \usepackage{commath}
    \doublespacing
    %opening
    \title{}
    
    %===> Formatting ===>
    \setlength{\parskip}{8pt}
    \setlength{\parindent}{20pt}
    %<=== Formatting <===
    
    
    \title{CSE 415 Homework 2 Part 1}
    \author{Chongyi Xu}
    
    \begin{document}
        
    \maketitle
    \begin{questions}
        \question For each of the following, determine which of the two relations “subsetof”
        or “element-of” is being represented, and reformulate the statement
        to make this clearer. The first one is done for you. If you find genuine
        ambiguity in a statement, justify each of the possible interpretations.
        \begin{itemize}
            \item Fido is a dog
            \\ Fido $\in$ dogs

            \item A parrot is a bird
            \\ parrot $\subset$ birds

            \item Polly is a parrot
            \\ Polly $\in$ parrots

            \item David Jones is a Jones
            \\ David Jones $\in$ Jones

            \item \"George Washington\" is a great name
            \\ George Washington $\in$ names

            \item Artificial Intellengence is a state of mind
            \\ Artificial Intellengence $\subset$ state of minds
        \end{itemize}

        \question For each of the following relations, state whether or not it is reflexive,
        whether or not it is symmetric, whether or not it is transitive, whether
        or not it is antisymmetric, and whether or not it is a partial order. For
        each example, let the set S on which the relation is defined be the set of
        elements mentioned in that example.
        \begin{itemize}
            \item {(a, a)}
            Reflexive, antisymmetric, symmetric and transitive. So it is partial order.
            \item {(a, b), (a, c), (b, c)}
            Antisymmetric only.
            \item {(a, b), (b, c)}
            Antisymmetric only.
            \item {}
            Reflexive, antisymmetric, symmetric and transitive. So it is partial order.            
        \end{itemize}
    \end{questions}
    \end{document}